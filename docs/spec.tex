\documentclass[a4paper,12pt]{report}
\usepackage[MeX]{polski}
%\usepackage[cm]{fullpage}
\usepackage{fontspec}
\usepackage{graphicx}
\usepackage{tabularx}
\usepackage{hyphenat}
\usepackage[pdfborder={0 0 0}]{hyperref}
%\usepackage{mathspec}
%\usepackage{MnSymbol}
%\usepackage{mathabx}

\linespread{1.5}

% Scale=MatchLowercase <- nie wiem co to robi, można dodać poniżej
\defaultfontfeatures{Mapping=tex-text}
\setmainfont[
  Ligatures={Common, Rare},
  Numbers={Monospaced,OldStyle},
  Scale=1.2
]{Linux Libertine O}
\setmonofont{Inconsolata}

\title{Wysypisko}
\author{Sławek Borzdyński\\Michał Charmas\\Kasia Krawczyk\\Szymon Witamborski}
\date{9 października 2011}

\begin{document}
\maketitle
\tableofcontents
\chapter{Koncepcja gry}

Gra ma być wariantem gry typu Tower Defence. Modyfikacja schematu polega na tym, że na arenie będzie wielu graczy współzawodniczących o to kto pozostanie przy życiu.

\section{Kluczowe zamysły}

Gracze lądują statkami kosmicznymi równo rozmieszczeni na okręgu. Na przeciwko każdego gracza znajduje się dziura w ziemi, z której wychodzą roboty-śmieciarze starające się rozebrać statki i zanieść zdobyte części na wysypisko znajdujące się na środku planszy. Gracze stawiają różnych typów zapory uniemożliwiające robotom dotarcie do statków. Wygrywa ostatni ostały gracz.

\section{Wykorzystane technologie}

OGRE, coś do serwera, Python ?

\section{Wymagania sprzętowe}

Gra ma działać na systemach Linux i Windows na sprzęcie ze średniej półki (ma działać płynnie na wbudowanych układach Intela).

\chapter{Historia}

Nastały ciężkie czasy dla Układu Słonecznego. Tak ciężkie, że nikt nie potrafi ustalić który jest rok\ldots

Najbardziej wartościowym i poszukiwanym towarem stały się części zamienne do statków kosmicznych. Niestety od tysięcy lat nikt ich nie produkuje. Wiedza jak je wytwarzać zniknęła w trakcie konfliktu atomowego na Ziemi.

Ostało się tylko jedno miejsce, Wysypisko, posiadające ten cenny towar. Strzegą je tępe roboty, Śmieciarze, których jedynym zadaniem i rozrywką jest rozbieranie przybyłych statków i gromadzenie części.

Wysypisko otoczone jest lądowiskiem dla śmiałków, jednak natychmiast znajdują się w potrzasku, ponieważ po wylądowaniu znajdują się bezpośrednio na drodze Śmieciarzy do Wysypiska.

Jednak, Śmieciarze są na tyle gościnni, że oddadzą części ale tylko jednemu z przybyłych gości.

\chapter{Rozgrywka}

Gracze są rozmieszczeni równomiernie wokół środka areny. Każdy gracz broni swojego statku ustawiając budowle utrudniające wrogim jednostkom dotarcie do niego. Rozgrywka składa się z etapów. Każdy etap zawiera dwie fazy: budowy i walki. Faza budowy trwa przez określoną ilość czasu. Gracze przygotowują się w niej do fazy walki poprzez rozmieszczanie budowli na arenie. Podczas fazy walki wrogie jednostki przemieszczają się z określonych punktów przy krawędzi areny w kierunku obiektów należących do graczy. Faza kończy się w momencie, kiedy wszystkie wrogie jednostki zostaną zniszczone lub dotrą do składowiska na środku areny.

\section{Etap budowy}

Gracz ustawia w wybranych przez siebie miejscach budowle. Może również naprawiać bądź ulepszać postawione wcześniej budowle. Każda z tych czynności kosztuje określoną liczbę części i trwa określony czas.

Budowle mają "punkty życia".

\subsection{Dostępne typy budowli}

\begin{description}
\item[wieżyczki niszczące] \hfill\\ ulepszalne na kilku poziomach, reszta do przemyślenia.
\item[bariery] \hfill
	\begin{description}
	\item[blokująca (rozbieralna)] \hfill\\ robot, który napotkał ją na drodze zatrzymuje i próbuje ją rozebrać zabierając jej punkty życia. Przechodzi po zniszczeniu przeszkody.
	\item[bariera zwalniająca wroga] \hfill\\ plama oleju (?), każda jednostka, która przez nią przejdzie zabiera ileś punktów życia.
	\item[bariera myląca (zmieniająca) ścieżkę wroga] \hfill\\ robot, który ją napotkał zmienia tor ruchu. bariera umiera po określonym czasie albo po określonej liczbie napotkanych robotów.
	\end{description}
\end{description}
\section{Etap walki}

Wrogowie atakują. Można budować j/w tylko, że wolniej.

\subsection{Rodzaje wrogów}

\begin{description}
\item[niszczyciel statków] \hfill\\ celem jest statek, rozwala tylko niezbędne przeszkody
\item[niszczyciel barier] \hfill\\ celem są przeszkody, przy braku przeszkód idą do statków.
\end{description}

\subsection{Możliwości gracza w trakcie walki}
\begin{itemize}
\item naprawa
\item ulepszanie
\end{itemize}

\section{Zakończenie rozgrywki}

Zostaje jeden gracz i odlatuje z potrzebnymi częściami, szczęśliwy.


\chapter{Interfejs użytkownika}

\chapter{Grafika}

\chapter{Muzyka}

\chapter{Do przemyślenia}

\begin{itemize}
\item kradziejowanie?
\item sposób zdybania graczy (serwer centralny z "pokojami" a może z przydzielaniem współzawodników FIFO?)
\end{itemize}

\end{document}
